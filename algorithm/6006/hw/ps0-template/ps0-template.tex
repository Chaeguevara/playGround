%
% 6.006 problem set 0 solutions template
%
\documentclass[12pt,twoside]{article}

\input{macros-sp20}
\newcommand{\theproblemsetnum}{0}

\title{6.006 Problem Set 0}

\begin{document}

\handout{Problem Set \theproblemsetnum}

\setlength{\parindent}{0pt}
\medskip\hrulefill\medskip

{\bf Name:} Heejin, Chae

\medskip\hrulefill

%%%%%%%%%%%%%%%%%%%%%%%%%%%%%%%%%%%%%%%%%%%%%%%%%%%%%
% See below for common and useful latex constructs. %
%%%%%%%%%%%%%%%%%%%%%%%%%%%%%%%%%%%%%%%%%%%%%%%%%%%%%

% Some useful commands:
% $f(x) = \Theta(x)$
% $T(x, y) \leq \log(x) + 2^y + \binom{2n}{n}$
% \ttt{code\_function}


% You can create unnumbered lists as follows:
% \begin{itemize}
%     \item First item in a list
%         \begin{itemize}
%             \item First item in a list
%                 \begin{itemize}
%                     \item First item in a list
%                     \item Second item in a list
%                 \end{itemize}
%             \item Second item in a list
%         \end{itemize}
%     \item Second item in a list
% \end{itemize}

% You can create numbered lists as follows:
% \begin{enumerate}
%     \item First item in a list
%     \item Second item in a list
%     \item Third item in a list
% \end{enumerate}

% You can write aligned equations as follows:
% \begin{align}
%     \begin{split}
%         (x+y)^3 &= (x+y)^2(x+y) \\
%                 &= (x^2+2xy+y^2)(x+y) \\
%                 &= (x^3+2x^2y+xy^2) + (x^2y+2xy^2+y^3) \\
%                 &= x^3+3x^2y+3xy^2+y^3
%     \end{split}
% \end{align}

% You can create grids/matrices as follows:
% \begin{align}
%     A =
%     \begin{bmatrix}
%         A_{11} & A_{21} \\
%         A_{21} & A_{22}
%     \end{bmatrix}
% \end{align}

\begin{problems}

\problem  % Problem 1

\begin{problemparts}
\problempart 
$A = {1,6,12,13,9}, B = {3,6,12,15} ,A \cap B = \{6,12\} $
\problempart 
$\lvert A \cup B \rvert = 7 $
\problempart % Problem 1c
$\lvert A - B \rvert = 3 $
\end{problemparts}

\problem  % Problem 2

\begin{problemparts}
\problempart % Problem 2a
\problempart % Problem 2b
\problempart % Problem 2c
\end{problemparts}

\problem  % Problem 3

\begin{problemparts}
\problempart % Problem 3a
\problempart % Problem 3b
\problempart % Problem 3c
\end{problemparts}

\problem  % Problem 4
Prove it for n = 1 and k and k+1 \\
\begin{align*}
  1^3 &= \left( \frac{(1)(1+1)}{2} \right)^2 \\
\sum k^3 &= \sum (k-1)^3 + k^3 \\
         &= \left( \frac{(k-1)(k)}{2} \right)^2 + k^3 \\
         &= \frac{(k-1)^2(k)^2 +4k^3}{4} \\
         &= \frac{(k+1)^2(k)^2}{4} \\
\end{align*}


\newpage
\problem  % Problem 5

\vfill
\problem  % Problem 6
Submit your implementation to {\small\url{alg.mit.edu}}.

\begin{lstlisting}
def count_long_subarray(A):
    '''
    Input:  A     | Python Tuple of positive integers
    Output: count | number of longest increasing subarrays of A
    '''
    count = 0
    ##################
    # YOUR CODE HERE #
    ##################
    return count
\end{lstlisting}

\end{problems}

\end{document}
