%
% 6.006 problem set 3 solutions template
%
\documentclass[12pt,twoside]{article}

\input{macros-sp20}
\newcommand{\theproblemsetnum}{3}

\title{6.006 Problem Set \theproblemsetnum}

\begin{document}

\handout{Problem Set \theproblemsetnum}

\setlength{\parindent}{0pt}
\medskip\hrulefill\medskip

{\bf Name:} Heejin Chae

\medskip

{\bf Collaborators:} Name1, Name2

\medskip\hrulefill

%%%%%%%%%%%%%%%%%%%%%%%%%%%%%%%%%%%%%%%%%%%%%%%%%%%%%
% See below for common and useful latex constructs. %
%%%%%%%%%%%%%%%%%%%%%%%%%%%%%%%%%%%%%%%%%%%%%%%%%%%%%

% Some useful commands:
%$f(x) = \Theta(x)$
%$T(x, y) \leq \log(x) + 2^y + \binom{2n}{n}$
% {\tt code\_function}


% You can create unnumbered lists as follows:
%\begin{itemize}
%    \item First item in a list
%        \begin{itemize}
%            \item First item in a list
%                \begin{itemize}
%                    \item First item in a list
%                    \item Second item in a list
%                \end{itemize}
%            \item Second item in a list
%        \end{itemize}
%    \item Second item in a list
%\end{itemize}

% You can create numbered lists as follows:
%\begin{enumerate}
%    \item First item in a list
%    \item Second item in a list
%    \item Third item in a list
%\end{enumerate}

% You can write aligned equations as follows:
%\begin{align}
%    \begin{split}
%        (x+y)^3 &= (x+y)^2(x+y) \\
%                &= (x^2+2xy+y^2)(x+y) \\
%                &= (x^3+2x^2y+xy^2) + (x^2y+2xy^2+y^3) \\
%                &= x^3+3x^2y+3xy^2+y^3
%    \end{split}
%\end{align}

% You can create grids/matrices as follows:
%\begin{align}
%    A =
%    \begin{bmatrix}
%        A_{11} & A_{21} \\
%        A_{21} & A_{22}
%    \end{bmatrix}
%\end{align}

% You can include images and PDFs as follows:
% \includegraphics[width=0.5\textwidth]{img.jpg}

\begin{problems}

\problem  % Problem 1

\begin{problemparts}
\problempart % Problem 1a
\problempart % Problem 1b
\end{problemparts}

\newpage

\problem  % Problem 2

\begin{problemparts}
\problempart % Problem 2a
\problempart % Problem 2b
\problempart % Problem 2c
\end{problemparts}

\newpage

\problem  % Problem 3

\begin{problemparts}
\problempart % Problem 3a
\problempart % Problem 3b
\problempart % Problem 3c
\problempart % Problem 3d
\end{problemparts}

\newpage

\problem  % Problem 4

\begin{problemparts}
\problempart % Problem 4a
\problempart % Problem 4b
\end{problemparts}

\newpage

\problem  % Problem 5

\begin{problemparts}
\problempart idea : Turn string into base 26 integer sequece. For example "abcdde" becomes "11121...0". Each position corresponding alphabet place.\\
 First prove build time is $O(|A| )$ \\
 Building a hash table from A takes $O(k)+n*O(1)$. For first k chunks, the initial frequency is established by iterating over k elements in A. Then by sliding 1 elements side by side it doesn't need to be rebuild. But by memorizing prevous first item, decrement the corresponding frequency and increment the one for the next added one. So by repeating this process, The overall time complexity is $O(k)+n*O(1)$ which $O(k)$ is thrown by first build and $n*O(1)$ for sliding windows
\problempart % Problem 5b
\problempart Submit your implementation to {\small\url{alg.mit.edu}}.
\end{problemparts}

\end{problems}

\end{document}
